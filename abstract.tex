
\chapter*{R�sum� et Mots Cl�s}
\subsubsection*{R�sum�}
SAIDAL est une Soci�t� par actions, au capital de 2 500 000 000 dinars alg�riens. Cr�� en 1982, le groupe couvre actuellement 40\% des besoins nationaux en m�dicaments ce qui le place en position de leader de l'industrie pharmaceutique en Alg�rie.\\[0.5\baselineskip]

Avec neuf sites de production, le groupe SAIDAL dispose d'un important parc d'�quipements industriels, ce qui impose un important effort de maintenance pour permettre � ces ces �quipements d'�tre dans le meilleur �tat afin d'obtenir un meilleur rendement en terme de qualit� et de quantit� des produits.\\[0.5\baselineskip]

Actuellement, la maintenance au sein du groupe se fait manuellement, avec notamment des support papier. Cette gestion rend difficile l'exploitation de l'information, les interventions ainsi que l'�laboration de diff�rents rapport quotidien et mensuelles et les statistiques relatifs � la fonction maintenance. Cette mauvaise gestion risque de compromettre le statut du groupe comme �tant leader national de l'industrie des m�dicaments.\\[0.5\baselineskip]

Dans ce contexte, nous nous sommes engag�, avec la direction des syst�mes d'information du groupe, dans un projet de r�alisation d'un syst�me automatique pour la gestion de la maintenance GMAO qui sera d�ploy� dans un premier temps au sein de l'unit� de production de Dar El Beida. Ce projet devrait permettre d'am�liorer largement la gestion des travaux de maintenance dans cette unit� et de r�duire sensiblement les co�ts des interventions.  


\subsubsection*{Mots Cl�s}
GMAO, Maintenance pr�ventive, Maintenance corrective, Intervention, Demande de r�paration, Ordre de travail, Service m�thodes, SAIDAL.

\newpage

\subsubsection*{Abstract}
SAIDAL is a joint stock company with a capital of 2.5 billion Algerian dinars. Created in 1982, the group currently covers 40 \% of national drug needs which means that it's the leader of pharmaceutical industry in Algeria.\\[0.5\baselineskip]

With nine production sites, SAIDAL the group has a large fleet of industrial equipment, which requires significant maintenance effort to allow to these equipements to be those in the best condition to obtain better performance in terms quality and quantity of products.\\[0.5\baselineskip]

Currently, maintenance within the group is done manually, particulary with papers. This management makes it difficult to use informations, interventions and the development of various daily and monthly report and statistics relating to the maintenance function. This mismanagement could compromise the group's status as national leader of the drug industry.\\[0.5\baselineskip]

In this context, we are committed, with management information systems of the group, in a project to build an CMMS that will be deployed initially in the unit production of Dar El Beida. This project is expected to greatly improve the management of maintenance work in this unit and significantly reduce intervention costs.\\[0.5\baselineskip]

??????? ?????????

\subsubsection*{Key words}
CMMS, Preventive maintenance, Corrective maintenance, Intervention, application works, Work order, service methods, SAIDAL.


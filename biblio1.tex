\begin{thebibliography}{MON00}
\expandafter\ifx\csname fonteauteurs\endcsname\relax
\def\fonteauteurs{\scshape}\fi

\bibitem[\bfseries AFN02]{ref02}
\bgroup\fonteauteurs\bgroup AFNOR\egroup\egroup{} :
\newblock Maintenance industrielle: Fonction maintenance. indice de classement:
  Fd x 60-000.
\newblock Disponible en ligne sur:
  \url{http://imis.angers.free.fr/site/IMG/pdf/FDX_60-000.pdf}, Mai 2002.
\newblock [Consult� le 20 Octobre 2015].

\bibitem[\bfseries BUC00]{web02}
Nicolas \bgroup\fonteauteurs\bgroup BUCHY\egroup\egroup{} :
\newblock La gestion de la maintenance assist�e par ordinateur et la
  maintenance des logiciels.
\newblock Disponible en ligne sur:
  \url{http://s3.amazonaws.com/publicationslist.org/data/gelog/ref-507/115.pdf},
  2000.

\bibitem[\bfseries dl11]{web09}
Minist�re de~\bgroup\fonteauteurs\bgroup l'Industrie\egroup\egroup{} :
\newblock Rapport sectoriel n�01: L'industrie pharmaceutique.
\newblock page~3, Janvier 2011.

\bibitem[\bfseries DRE08]{web03}
Vincent \bgroup\fonteauteurs\bgroup DRECQ\egroup\egroup{} :
\newblock La gmao en quelques lignes.
\newblock Disponible en ligne sur:
  \url{http://www.conseilorga.com/Documents/octobre%202008%20-%20La%20GMAO%20en%20quelques%20lignes.pdf},
  Octobre 2008.
\newblock [Consult� le 20 Octobre 2015].

\bibitem[\bfseries AA13]{web07}
Ouardia \bgroup\fonteauteurs\bgroup ATEK\egroup\egroup{} et Farida
\bgroup\fonteauteurs\bgroup ADMANE\egroup\egroup{} :
\newblock Introduction aux syst�mes d'information.
\newblock Cours de syst�me d'information (2CPI), Ecole nationale superieure
d'informatique, ALGER, 2013.

\bibitem[\bfseries KAF01]{ref04}
H�di \bgroup\fonteauteurs\bgroup KAFFEL\egroup\egroup{} :
\newblock {\em Contribution La maintenance distribu�: concepts, �valuation et
  mise en \oe uvre}.
\newblock Th\`ese de doctorat, universit� laval qu�bec, Octobre 2001.

\bibitem[\bfseries LAC11]{web08}
Mohamed-Ch�rif \bgroup\fonteauteurs\bgroup LACHICHI\egroup\egroup{} :
\newblock Lotfi benbahmed, pr{\'e}esident du conseil national de l'ordre des
  pharmaciens, au forum de "libert{\'e}".
\newblock {\em "LIBERT{\'E}"}, page~6, 16 mars 2011.

\bibitem[\bfseries MEC04]{ref06}
Bernard \bgroup\fonteauteurs\bgroup MECHIN\egroup\egroup{} :
\newblock {\em Documentation de la fonction maintenance}.
\newblock Ed. Techniques Ing�nieur, 2004.

\bibitem[\bfseries MON00]{ref01}
Fran�ois \bgroup\fonteauteurs\bgroup MONCHY\egroup\egroup{} :
\newblock {\em Maintenance: M�thodes et organisations}.
\newblock Dunod \'edition, 2000.

\bibitem[\bfseries Oph08]{ref05}
D.~\bgroup\fonteauteurs\bgroup Ophelie\egroup\egroup{} :
\newblock Gestion de la maintenance assist�e par ordinateur.
\newblock {\em Publications Oboulo. com}, 2008.

\bibitem[\bfseries RUF11]{web04}
Jean-Michel \bgroup\fonteauteurs\bgroup RUFFIN\egroup\egroup{} :
\newblock pourquoi une gmao.
\newblock Disponible en ligne sur:
  \url{www.jmr-gmao.com/FTPJMR/Pourquoi%20une%20GMAO.doc}, 2011.
\newblock [Consult� le 20 Octobre 2015].

\bibitem[\bfseries SML97]{ref03}
Marc \bgroup\fonteauteurs\bgroup SAINT-MARSEILLE\egroup\egroup{} et Jean-Bruno
  \bgroup\fonteauteurs\bgroup LAPOINTE\egroup\egroup{} :
\newblock {\em La gestion des �quipements: vers l'entretien pr�ventif}.
\newblock Association paritaire pour la sant� et la s�curit� du travail
  secteur fabrication de produits en m�tal et de produits �lectriques
  \'edition, 1997.

\bibitem[\bfseries weba]{web05}
Comparatif des logiciels gmao.
\newblock Disponible en ligne sur:
  \url{http://gii.polytech.up.univ-mrs.fr/deuterium/page_guide.php?num_page=66}.
\newblock [Consult� le 20 Octobre 2015].

\bibitem[\bfseries webb]{web06}
Gestion de la maintenance: les logiciels de gmao.
\newblock Disponible en ligne sur:
  \url{www.mesures.com/pdf/old/778GA_logiciel_GMAO.pdf}.
\newblock [Consult� le 20 Octobre 2015].

\bibitem[\bfseries webc]{web10}
Site officiel du groupe saidal.
\newblock Disponible en ligne sur: \url{https://www.saidalgroup.dz}.
\newblock [En ligne: Consult� le 20 Octobre 2015].

\bibitem[\bfseries web14]{web01}
Introduction � la maintenance.
\newblock Disponible en ligne sur:
  \url{http://www.technologuepro.com/maintenance-industrielle/chapitre-4-la-documentation-enmaintenance.pdf},
  2014.
\newblock [Consult� le 20 Octobre 2015].

\bibitem[\bfseries ROQUES VAL, 2004]{ref07}
Pascal \bgroup\fonteauteurs\bgroup ROQUES\egroup\egroup{} and
Franck  \bgroup\fonteauteurs\bgroup VALLEE\egroup\egroup{} :
\newblock {\em UML 2 en action de l'analyse des besoins � la conception J2EE}.
\newblock 3�me �dition Ed. EYROLLES, 2004.

\bibitem[\bfseries ROQUES VAL, 2007]{ref08}
Pascal \bgroup\fonteauteurs\bgroup ROQUES\egroup\egroup{} and
Franck  \bgroup\fonteauteurs\bgroup VALLEE\egroup\egroup{} :
\newblock {\em UML 2 en action de l'analyse des besoins � la conception J2EE}.
\newblock 4�me �dition Ed. EYROLLES, 2007.

\bibitem[\bfseries ROQUES, 2008]{ref09}
Pascal \bgroup\fonteauteurs\bgroup ROQUES\egroup\egroup{} :
\newblock {\em UML 2 Mod�liser une application web }.
\newblock 4�me �dition Ed. EYROLLES, 2008.

\bibitem[\bfseries MB11]{web11}
Mohammed Amine   \bgroup\fonteauteurs\bgroup MOSTEFAI\egroup\egroup{} et Sofiane  
\bgroup\fonteauteurs\bgroup BATATA\egroup\egroup{} :
\newblock Expression des besoins.
\newblock Cours d'IGL (Introduction de G�nie Logiciel) (1CS), Ecole nationale superieure
d'informatique, ALGER, 2011.

\bibitem[\bfseries webd]{web12}
Choix d'une architecture trois-tiers
\newblock Disponible en ligne sur:
\url{http://cedric.babault.free.fr/rapport/node4.html}.
\newblock [Consult� le 26 Avril 2016].



\bibitem[\bfseries AUDIBERT, 2009]{ref11}
Laurent \bgroup\fonteauteurs\bgroup AUDIBERT\egroup\egroup{} :
\newblock {\em UML 2 De L?apprentissage � La Pratique }.
\url{http://laurent-audibert.developpez.com/Cours-UML/?page=introduction-modelisation-objet#L1-4-3-f},
2009.
\newblock [Consult� le 17 Mai 2016].



\end{thebibliography}
\chapter{\sc R�alisation et s�curit� de la solution }
\section{Introduction}
Nous consacrons ce dernier chapitre � la pr�sentation de la phase de r�alisation de notre
projet. La partie r�alisation est tout aussi importante que la conception. En effet, c'est au
cours de celle-ci que nous concr�tisons notre travail. Ceci par l'impl�mentation d'un prototype de la solution pr�sent�e dans les pr�c�dents chapitres.
Dans la suite de ce chapitre, nous d�crivons, d'une part, les technologies utilis�es, ainsi que l'aspect s�curit� en pr�sentant les qualit�s s�curitaires de notre solution et les diff�rentes pr�cautions prises. D'autre part, nous pr�senterons certaines fonctionnalit�s � travers des captures d'�crans des interfaces de notre application.
\newpage

\section{Les technologies de d�veloppements utilis�es}
\subsection{Choix du SGBD}

\subsection{Choix du l'IDE}
\subsection{Choix du serveur Web}
\subsection{Choix du langage}
\subsection{N�cessit� d'un Framework}


\section{La s�curit� du syst�me}


\section{Pr�sentation du prototype r�alis�}
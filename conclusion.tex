\chapter*{Conclusion g�n�rale et perspectives}

Dans ce m�moire nous avons pr�sent� le travail r�alis� dans le cadre de notre projet de fin
d'�tudes, et dont l'objectif �tait la conception et la r�alisation d'un syst�me d'information pour la gestion de la maintenance au sein du groupe SAIDAL. Cette derni�re �prouve des besoins dans la gestion actuelle de la maintenance. Cette derni�re, �tant une gestion manuelle, ne permet pas d'effectuer des intervention en temps voulu, ne permet pas une bonne planification des interventions pr�ventive et encore moins une bonne tra�abilit� de l'information. L'entreprise � jug� utile de doter son service de maintenance d'une application de gestion de la maintenance assist� par ordinateur.\\[0.5\baselineskip]
Une premi�re �tape �tait d'effecteur une �tude bibliographique sur la maintenance afin de d�couvrir ce domaine. Nous avons pris connaissance des concepts cl� de la maintenance, � savoir , les differents type de maintenance existants (corrective, pr�ventive), les gammes de maintenance, le PMP, ainsi que les objectifs de la fonction maintenance. Cette �tude nous a aussi permet d'appr�hender les concepts de la GMAO et les fonctionnalit�s qu'elle offre et son apport dans le domaine de la maintenance.\\[0.5\baselineskip]
La deuxi�me �tape d'effectuer une �tude de l'existant. �a a permet de se familiariser avec les diff�rents proc�dures de maintenance au sein de SAIDAL, et ainsi de d�tecter les points faibles de la gestion actuelle et ainsi proposer une solution optimal et propre leurs processus de maintenance.\\[0.5\baselineskip]
La prochaine �tape, 


